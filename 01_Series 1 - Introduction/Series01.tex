\documentclass[11pt]{article}

\setcounter{tocdepth}{3}
\setcounter{secnumdepth}{3}

\usepackage{comment} % enables the use of multi-line comments (\ifx \fi) 
\usepackage{lipsum} %This package just generates Lorem Ipsum filler text. 
\usepackage[a4paper,margin=1.5cm]{geometry}
\usepackage[utf8]{inputenc}
\usepackage[ngerman]{isodate}
\usepackage{gensymb}
\usepackage{graphicx}
\usepackage{booktabs}% http://ctan.org/pkg/booktabs
\usepackage{makecell}
\usepackage{tabularx}
\usepackage[table]{xcolor}
\usepackage{array}
\usepackage{wrapfig}
\usepackage{subcaption}
\usepackage{csquotes}
\usepackage{lscape}
\usepackage{afterpage}
\usepackage{geometry}
\usepackage{listingsutf8}
\usepackage{chngcntr}
\usepackage{multicol}
\usepackage{xcolor}
\usepackage{pifont}
\usepackage{outlines}
\usepackage{amsmath}
\usepackage{amssymb}
\usepackage{breqn}
\usepackage{textcomp}
\usepackage{bm}
\usepackage{caption}
\usepackage{enumitem}
\usepackage{hyperref}
\usepackage{mdframed}
\usepackage{scalerel}
\usepackage{stackengine}

\newmdtheoremenv{theorem}{Theorem}

\counterwithin{figure}{section}

\AtBeginDocument{\counterwithin{lstlisting}{section}}

\geometry{a4paper, margin=1in}

\renewcommand*{\thead}[1]{\bfseries #1}
\newcommand{\code}[1]{\texttt{#1}}
\def\doubleunderline#1{\underline{\underline{#1}}}

\newcommand\equalhat{\mathrel{\stackon[1.5pt]{=}{\stretchto{%
    \scalerel*[\widthof{=}]{\wedge}{\rule{1ex}{3ex}}}{0.5ex}}}}

\DeclareMathSizes{12}{30}{16}{12}

\definecolor{lightgray}{rgb}{.9,.9,.9}
\definecolor{darkgray}{rgb}{.4,.4,.4}
\definecolor{purple}{rgb}{0.65, 0.12, 0.82}
\definecolor{darkgreen}{rgb}{0.05,0.56,0.06}


\lstset{frame=tlrb,
    language=python,
    captionpos=b,
    aboveskip=3mm,
    belowskip=3mm,
    showstringspaces=false,
    columns=flexible,
    basicstyle={\small\ttfamily},
    numbers=left,
    numberstyle=\tiny\color{gray},
    keywordstyle=\color{blue},
    commentstyle=\color{violet},
    stringstyle=\color{darkgreen},
    breaklines=true,
    breakatwhitespace=true,
    tabsize=3,
    literate=%
    {Ö}{{\"O}}1
    {Ä}{{\"A}}1
    {Ü}{{\"U}}1
    {ß}{{\ss}}1
    {ü}{{\"u}}1
    {ä}{{\"a}}1
    {ö}{{\"o}}1
}


\begin{document}

\title{Optimization HS19 - Serie 01}
\author{Pascal Baumann}
\maketitle

\graphicspath{{./img/}}

\section{Exercise 2}
\textit{The construction of a lecture time-table can be mathematically formulated (simplified) as an assignment of a lecturer, a time slot, and a room to every lecture.}

\textit{Find an upper bound for the number of possible time-tables if there are 500 lectures per week, 100 lecturers, 40 time slots per week, and 15 rooms.}

\vspace{1em}

For every lecture there are $l$ lecturers, $t$ time slots and $r$ rooms. Or in a formulaic form $l\cdot t\cdot r$. A naïve approach (ignoring the conflicts of multiple lectures in the same room, given by the same lecturer at the same time) would give us an upper bound of
\begin{equation*}
    UpperBound = (l\cdot t \cdot r)^{500}
\end{equation*}
\begin{equation*}
    UpperBound = (100\cdot 40\cdot 15)^{500} \approx 1.1902\cdot 10^{2389} 
\end{equation*}

Given the enumeration time of 1 million time tables per second, this task woud take our computer:

\begin{equation*}
    \frac{1.1902\cdot 10^{2389}\text{ timetables}}{10^6\frac{\text{timetables}}{\text{second}}} = 1.1902\cdot 10^{2383}\text{ seconds} \approx 3.774\cdot 10^{2375}\text{ years}
\end{equation*}


\end{document}
